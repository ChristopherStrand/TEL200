\documentclass[a4paper,12pt]{article}

\usepackage[utf8]{inputenc}  
\usepackage{graphicx}       
\usepackage{geometry}
\usepackage{parskip}

\geometry{a4paper, margin=2.5cm} 

\title{TEL200 - ABB RobotStudio and YuMi Project}
\author{J\o rgen Asmundvaag \\ Ludvik H\o iberg-Aslaksen \\ Christopher Strand}
\date{March 2025}

\begin{document}

\maketitle

\newpage
\section{Abstract}
\section{Introduction}
Industrial robotics plays an increasingly vital role in enhancing productivity, precision, and efficiency within modern manufacturing processes. ABB RobotStudio and RAPID provides a powerful platform for simulating and optimizing robotic operations in a virtual environment prior to physical implementation.

In this project, we use ABB's RobotStudio simulation environment to develop, simulate, and deploy two different applications as described in the task description. 

The first application involves precisely picking up objects and moving them to another designated location at the click of a button, with additional safety functionality integrated through an emergency button.

The second application is the YuMi challenge where we were given an open task to test the capabilities of the YuMi robot. Inspired by the 18th century famous machine that fascinated audiences across Europe, "The Turk". As The Turk we won´t either play fully autonomous and will use scholar´s mate to make it possible to finish within the given time frame. This  will provide an understanding of the functionality of YuMi robot and it´s capabilities doing precise tasks.

The report provides detailed insights into the methods, procedures, and results achieved through these applications, emphasizing practical experiences and theoretical connections drawn from the course syllabus.

\section{Method}
\section{Results}
\section{Discussion}



\end{document}

